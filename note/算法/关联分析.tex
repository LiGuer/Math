* 关联分析
	\Situation
		存在一个0/1表格, 每一行对应一个记录, 每一列对应一个项, 表中每一个0/1值表示该项是否存在于该事物中.

		- 项集: 包含一些项的集合.

		- 关联规则
			$x \to y$
			其中, $x \cup y = \emptyset$
			即 关联规则表征两个项之间可能存在很强的关联.

		- 度量指标: 支持度、置信度
			对于项集$X, Y$
			$
				support(X \to Y) &= \/{number(X \cup Y)}{number(T)}  \tag{支持度}
				confidence(X \to Y) &= \/{number(X \cup Y)}{number(X)}  \tag{支持度}
				number(X) &= number({t_i | X \subseteq t_i, t_i \in T})
			$
			支持度表征数据集中包含该项集的记录所占的比例.
			- $T = \{t_1, ... ,t_N\}$ 所有记录的集合, $t_i$是表格第i行的记录.

		- 频繁集
			关联规则的两个项集的并集, 且其关联度大于等于给定的阈值. 即 频繁集表征经常出现在一起的项的集合.

	\Problem
		寻找关联规则, 即关联度、置信度都大于等于给定的阈值.
		$
			support(X \to Y) &≥ minsupport
			confidence(X \to Y) &≥ minconfidence
		$

	\Property
		- 项集频繁, 则其子集频繁$ <=> $项集不频繁, 则其超集不频繁.
		- 若规则$X \to Y - X$低于置信度阈值, 则对于$X$子集$X'$, 规则$X' \to Y - X'$也低于置信度阈值
		- 频繁项集生成的方法:
			- $F_k = F_{k-1} × F_1$
			- $F_k = F_{k-1} × F_{k-1}$

	\Algorithm
		- Apriori
			\Note
				- 频繁项集每一项各不相同,  每一项内部排列有序.

			- 步骤
				- 对于K = 1 : N项的项集, 迭代
					- 频繁项集生成
						对于K项项集
						- 频繁项集子集生成. 生成K项所有可以组合的集合. 
							eg.(frozenset({2, 3}), frozenset({3, 5})) -> (frozenset({2, 3, 5}))
						- 保存满足支持度阈值的集合.
					-  关联规则生成
						对不同长度(K)的频繁项集依次分析
						- 频繁项集只有两个元素{AB}, 直接计算置信度P(A→B),P(B→A)
						- 频繁项集超过两个元素{ABC...}, 依次计算置信度P(AC...→B)
						- 保存满足目标置信度的关联规则.
