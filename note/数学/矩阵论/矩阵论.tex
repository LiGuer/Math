* 矩阵论
	* 线性空间
		\def
			一个带有加法和数乘的非空集合, 且满足下列条件,
			- 加法封闭 $x+y \in V$
			- 数乘封闭 $k x \in V$
			- 存在零元 $x+0=x$
			- 存在负元 $x+(-x) = 0$
			- $1x = x$
			- 交换律 $x+y = y+x$
			- 分配律 $(k+l)x = kx+lx$
			- 加法结合律 $x+(y+z) = (x+y) +z$
			- 数乘结合律 $k(Lx) = (kl)x$
		\Property
			* 线性无关/线性相关
				$\nexists / \exists\ a ≠ 0 => x = \sum_{i=1}^n a_i x_i = 0$ 
			* 维数
				\def 
					线性空间中, 线性无关向量组所含向量最大的个数.
			* 基
				\def
					$\forall x \in V, and\ x = \sum_{i=1}^n a_i x_i$
					基是一个线性无关的向量组$X = (x_1, ... , x_n)$, 且线性空间中所有向量都是该向量组的线性组合.$x_i$ 基向量, $a_i$ 坐标.
				\Property 
					- 基变换
						$Y = X C$
						新旧基之间的变换矩阵.
						\Property
							基变换矩阵是非奇异矩阵.
					- 坐标变换
						$a_x = C a_y$
						\Proof
							$v = X a_x = Y a_y = X C a_y => a_x = C a_y$
					* 张成
						$Span(x_1,...,x_n) = \{x | x = \sum_{i=1}^n a_i x_i\}$ 
						线性空间由基向量给出的一种表示.
		* 线性子空间
			\def
				线性空间中的一个非空集合, 且对线性运算的封闭.
				- 加法封闭 $x,y\in V_1 ,\qu x+y \in V_1$
				- 数乘封闭 $x \in V_1, k x \in V_1$
		\Example
			* 内积空间
				\def
					定义了内积的空间.
					* 内积
						\def
							满足下列条件的运算,
							- 交换律:$<x, y>=\overline{<y, x>}$
							- 分配律:$<x, y+z> = <x , y> + <x, z>$
							- 齐次性:$<k x, y> = k <x, y>$
							- 非负性:$<x,x> ≥ 0$, 当且仅当 $x = 0, <x,x> = 0$
						\Property
							* 正交
								\def
									$<x,y> = 0$
									正交指两个向量的内积为零. 
							* \Theorem Cauchy-Бyнияковскнй不等式
								$|<x, y>| ≤ |x| |y|$
	* 线性变换
		\def
			$T(k x + l y) = k(T x) + l(T y)$
			线性空间$V$到自身的一类映射$T$, 对于所有$x \in V$都有唯一的$y \in V$与之对应, 且满足线性条件.
			* 线性变换矩阵
				\def
					$T X = X A  \qu; X = [x_1, ... , x_n]$ 
					线性变换由基的矩阵给出的一种表示.
				\Property
					- 运算 
						- $(T_1 + T_2) X = X (A + B)$
						- $(k\ T_1) X = X (k\ A)$
						- $(T_1 T_2) X = X AB$
						- $T_1^{-1} X = X A^{-1}$
		\Property
			* 值域 
				\def
					$Range(T)=\{T x | x \in V\}$
					线性空间中, 所有向量在线性变换后的结果的集合, 即 线性变换后的线性空间. 
				\Property
					* 秩
						\def
							$rank(A) = \dim Range(A) = \dim Range(A^T)$
							变换后的空间的维数, 即 值域的维数.
			* 零空间
				\def
					$Null(T) = \{x | T x = 0\}$
					线性空间中, 所有在线性变换为零向量的原向量的集合. 
				\Property
					$\dim V = \dim Range(A) + \dim Null(A)$
					变换前线性空间维数 = 值域维数 + 零空间维数. 
			* 不变子空间
				\def
					$\forall x \in V_1, V_1 \subseteq V, T x \in V_1$
			* 特征值、特征向量
				\def
					$T x = λ x$
					$x$ 特征向量, 是线性变换前后方向不改变的向量;
					$λ$ 特征值, 是特征向量在线性变换后长度变化的倍率.
				\Property
					* 特征多项式
						$\varphi(λ) = |λ I - A| = λ^n + a_1 λ^{n-1} + ... + a_{n-1} λ + a_n$
					* \Theorem Hamilton-Cayley定理
						$\varphi(A) = A^n + a_{1} A^{n-1}+ ... +a_{n-1} A + a_n I = 0$
						矩阵是其特征多项式的根.
			* 广义逆
				\def
					满足以下方程的解,
					$ \{\mb
						A X A = A
						X A X = X
						(A X)^H = A X
						(X A)^H = A X
					\me\right.$
					列满秩 $A^+ = (A^H A)^{-1} A^H$
					行满秩 $A^+ = A^H (A A^H)^{-1}$
				\Property
					* $rank(A) = rank(A^H A) = rank(A A^H)$
					* 满秩分解算广义逆 $A^+ = G^H (F^H A G^H)^{-1} F^H$
			* 相似
				\def
					$\exists \text{非奇异矩阵}P => B = P^{-1} A P$, 则$A$与$B$相似, 记作$A ~ B$.
				\Property 
					- $A ~ A$ 
					- $A ~ B <=> B ~ A$ 
					- $A ~ B, B ~ C <=> A ~ C$
					- 相似矩阵特征值、特征向量相同.
					- 相似矩阵迹相同.
			- 不同基下线性变换矩阵的转换
				$A_Y = C^{-1} A_X C \qu; Y = C X$
				\Proof
					$
						T Y = Y A_Y		 \tag{定义} 
						T X C = X C A_Y	 \tag{ Y = X C }
						X A_X C = X C A_Y   \tag{ T X = X A_X }
						A_X C = C A_Y
						A_Y = C^{-1} A_X C
					$
		\Example
			* 恒等变换
				\def $T x = x \qu ;(\forall x \in V)$
			* 零变换
				\def $T x = 0 \qu ;(\forall x \in V)$
			* 正交变换
				\def
					$<x, x> = <T x, T x>$
					内积空间中, 保持任意向量的长度不变的线性变换.
					正交矩阵:
						$A A^T = I$
						$A A^H = I$
				* 初等旋转变换
					\def
						初等旋转变换矩阵:
						$T_{ij} = (\mb
							\. I \\ & cosθ|_{(i,i)}&  & \sinθ|_{(i,j)} \\ & & \. I \\ & -\sinθ|_{(j,i)} & & \cosθ|_{(j,j)} \\ & & & & \. I
						\me)$
				* 初等反射变换
					\def
						$y = H x = (I - 2 e_2 e_2^T) x$
						\Proof
							$
								x - y = e_2 · (e_2^T x)
								=> y = (I-2 e_2 e_2^T) x
							$
			* 对称变换
				\def
					$<T x, y> = <x, T y>$
					对称矩阵:
						$A^T = A$
						$A^H = A$
			* 投影变换
				\def
					令线性空间分为不交的子空间L,M, 投影变换是将线性空间沿M到L的投影的变换.
					投影矩阵: 
						$P_{L,M} = (X & 0)\ (X & Y)^{-1}$
				* 正交投影变换
					\def
						设线性空间的子空间L, 将线性空间沿$L_\bot$到L的投影的变换, 称投影变换.
						正交投影矩阵:
							投影后子空间的基 $X = (x_1, ... , x_r) ,$ 则正交投影矩阵 $P_L = X(X^H X)^{-1}X^H$.
			* 斜切变换
				\def
					斜切变换矩阵: 
						单位矩阵的第(i,j)个元素改为斜切比率 $a_{ij}$
			* 缩放变换
				\def
					缩放变换矩阵:
					$T = (\mb dx_1 \\ & dx_2 \\ & & \ddots \\ & & & dx_n \me)$
					
	* 范数
		* 向量范数
			\def
				一类函数, 且满足
				- 非负性, $||A|| ≥ 0$, 当且仅当$A = 0, ||A|| = 0$
				- 齐次性, $||k A|| = |k| ||A||$
				- 三角不等式, $||A + B|| e ||A|| + ||B||$
			\Example 
				* $p$-范数 $||x||_{p}=(\sum_{i=1}^{n}|x_i|^p)^{1 / p}$
				* $∞$-范数 $||x||_∞ = \max|x_i|$
				* 椭圆范数 $||x||_A=(x^T A x)^{1/2}$
		* 矩阵范数
			\def
				一类函数, 且满足
				- 非负性, $||A|| ≥ 0$, 当且仅当$A = 0, ||A|| = 0$
				- 齐次性, $||k A|| = |k| ||A||$
				- 三角不等式, $||A + B|| e ||A|| + ||B||$
				- 相容性, $||A B|| e ||A||\ ||B||$
			\Example 
				* $||A||_{m_1} = \sum_{i,j} |a_{ij}|$
				* $||A||_{m_2} = (\sum_{i,j} a_{ij}^2)^{1/2}$
				* $||A||_{m_∞} = n·\max_{i,j}|a_{ij}|$
				* 列和范数 $||A||_1	  = \max_j \sum_i |a_{ij}|$
				* 行和范数 $||A||_∞ = \max_i \sum_j |a_{ij}|$
				* 谱范数   $||A||_2 = \sqrt{\max\ λ_i} \qu ,(λ_i)$为$A^H A$特征值.
		\Note
			矩阵范数, 向量范数相容: $||A x||_V \le ||A||_M ||x||_V$

	* 矩阵分解
