* 优化问题
	\Problem
		$
			\min \qu& f_0(x)  \tag{目标函数}
			s.t. \qu& f_i(x) ≤ 0  \tag{不等式约束}
						& h_i(x) = 0  \tag{等式约束}
		$
		优化问题的最优解: $p^*= \inf \{f_{0}(x) | f_i(x) ≤ 0, h_i(x) = 0 \}$

	\Include
		* 可行性问题
			\Problem
				$
					\min \qu& x 
					s.t. \qu& f_i(x) ≤ 0
								& h_i(x) = 0
				$
				若目标函数恒等于零, 则最优解是0 (可行集非空) 或$\infty$ (可行集为空集).

	* 凸优化问题
		\Problem
			$
				\min \qu& f_0(x) \tag{$f_0$为凸} 
				s.t. \qu& f_i(x) ≤ 0 \tag{$f_i$为凸} 
					& a_i^T x = b_i  \tag{仿射函数} 
			$

		\Algorithm{解凸优化问题}

		\Include
			* 线性规划
				\Problem
					$
						\min \qu& c^T x + d
						s.t. \qu& G x ⪯ h
							& A x = b
					$
					目标函数和约束函数都是仿射的优化问题.
					- 标准形式线性规划
						$
							\min \qu& c^T x
							s.t. \qu& A x = b
								& x ⪰ 0
						$
					- 不等式形式线性规划
						$
							\min \qu& c^T x
							s.t. \qu& A x ⪯ b
						$

				\Note
					- 可行域是多面体, 等位曲线是与向量$c^T$正交的超平面, 最优解是多面体中在$-c^T$方向最远的顶点.
					- 若线性规划问题存在两个最优解, 则其必然存在无穷多个最优解. 
				
				\Algorithm{线性规划转换为标准形式}
					- 步骤
						- 为不等式引入松弛变量
							$
								\min \qu& c^T x + d
								s.t. \qu& G x + s = h
									& A x = b
									& s ⪰ 0
							$
						- 将x表示为两个非负变量的差$x = x^+ - x^-, x^+ ⪰ 0, x^- ⪰ 0$, 代入即可转换为标准形式.
							$
								\min \qu& c^T x^+ - c^T x^- + d
								s.t. \qu& G x^+ - G x^- + s = h
									& A x^+ - A x^- = b
									& s ⪰ 0, x^+ ⪰ 0, x^- ⪰ 0
							$

			* 线性分式规划
				\Problem
					$
						\min \qu& \frac{a^T x+ b}{c^T x + d}
						s.t. \qu& Gx ⪯ 0
							& Ax = b
					$
					该问题可以等价转化为线性规划.

			* 二次规划
				\Problem
					$
						\min \qu& \frac{1}{2} x^T P x + q^T x + r
						s.t. \qu& Gx ⪯ 0
							& Ax = b
					$
				\Example
					最小二乘法 $\min ||Ax + b||_2^2$

			* 二次约束二次规划
				\Problem
					$
						\min \qu& \frac{1}{2} x^T P_0 x + q_0^T x + r_0
						s.t. \qu& \frac{1}{2} x^T P_i x + q_i^T x + r_i ⪯ 0
							& Ax = b
					$

			* 二次锥规划
				\Problem
					$
						\min \qu& f^T x
						s.t. \qu& ||A_i x + b_i|| ≤ c_i^T + d_i
							& Fx = g
					$

			* 几何规划
				\Problem
					$
						\min \qu& f_0(x)
						s.t. \qu& f_i(x) ≤ 1
							& h_i(x) = 1
					$
					自然形式不是凸的, 但可通过变换转换为凸优化问题.
					
			* 半正定规划
			\Note
				$二次锥规划 \supset \{二次规划 \supset \{ 线性规划 \} , 二次约束二次规划\}$
					
	* 整数规划
		\Problem
			优化问题中存在变量只能取整数的规划.

		\Algorithm
			- 分枝定界法
				求解纯整数或混合的整数规划问题.

		* 混合整数规划
			\Problem
				优化问题中既有连续变量, 又有整数变量的规划.

		* 0-1规划
			\Problem
				优化问题中变量仅取值0或1的规划.

	* 非凸优化问题
	