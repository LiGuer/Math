* 图
	\Define
		图是一个有序对 $(V, E)$
			- $V$ 点集合.
			- $E$ 边集, $E = \{(v_i, v_j) | v_i, v_j \in V\}$
			- 边权, 一个映射 $f: E \to \mathbb R$ 

		- 有向图
			边集元素$(v_i, v_j)$有序
			
		- 无向图
			边集元素$\{v_i, v_j\}$无序

	\Property
		- 表示方法
			- 邻接矩阵
			- 邻接链表

		- 同构

		- 连通性

		* 最短路径
			\Problem

			\Algorithm
				- Dijkstra 算法 (贪心)
					
					- 时间复杂度 $O(V^2)$

				- Floyd 算法 (动态规划)
					$d(i,j) = \min\{ d(i,k) + d(k,j) , d(i,j) \}  \tag{状态转移方程}$
					- 时间复杂度 $O(V^3)$
						空间复杂度 $O(V^2)$

		* 最小生成树
			\Problem

			\Algorithm
				- Prim 算法
					- 原理	
						按点贪心, 每次加入已搜索点集u的最短边(u,v), 其中v不属于已搜索点集的点v
					- 时间复杂度 $O(E·logV)$

				- Kruskal 算法
					- 原理
						按边贪心
					- 时间复杂度 $O(E·logV)$

		* 网络最大流
			\Problem

			\Algorithm
				- Dinic 算法
					- 原理
						贪心 + "反悔"机制

		* 商旅问题
			\Problem
				求遍历所有给定点的最短闭合路径.

	\Include
		* 二分图
			\Define
				$(X, Y, E)$
				- $X, Y \subset V, X \cup Y = V$
					图的点集分为不相交的两个子集$X, Y$
				- $E = \{(x_i, y_j) \ |\ x_i \in X, y_j \in Y\}$
					边只存在于点集$X, Y$之间, 而不存在于其内部.

			\Property
				- 匹配: 二分图的一个子图, 且子图的边集中任意两条边都不依附于同一顶点;
					设$M$是二分图$G$的子图, 则
					$\forall e_i, e_j \in E_M, e_i≠e_j, then\ e_i(1) ≠e_j(1), e_i(2) ≠e_j(2)$

					- 最大匹配: 边数最多的匹配 $\arg\max_{M \subseteq G} \qu number(E_M)$
					- 完美匹配: 所有点都在匹配的边上.

					- 求最大匹配
						\Problem
							
						\Algorithm
							- Hungarian 算法

		* 树、森林
			\Define
				- 森林: 无环路的无向图.
				- 树: 连通的无环路的无向图.

			\Property
				- 树的任意两点之间, 存在一条唯一的简单路径.
				- 树, $number(E) = number(V) - 1$
				- 树中, 删去一条边, 图变得不连通; 加上一条边, 图会出现环路.

			\Include
				* 二叉树
					\Define
						一棵树, 且满足有且只有一个根节点, 除根节点外的每个节点都有一个无向边连着前向, 所有节点最多只有2个无向边连着后向节点.

						- 根节点: 没有前向节点的节点, 且一棵树中有且只有一个.
						- 叶节点: 没有子节点的节点.
						- 左子树, 右子树: 某个节点的左(/右)子节点后继的子树.
						- 深度: 节点到根节点的简单路径中, 边的个数.

					\Property
						- 遍历
							- 先序遍历

							- 中序遍历
							- 后序遍历

					\Include
						* 完全二叉树
							\Define
								所有叶节点高度都相同的二叉树.

							\Property
								- 深度为h的节点数量 $2^h$
									非叶节点数量 $2^h - 1$
									叶节点数量 $2^h$

									\Proof
										$\sum_{i=0}^h 2^i = \/{1 - 2^h}{1 - 2} = 2^h - 1  \tag{等比数列求和}$

		* 有向无环图
			\Define
				无环路的有向图.

			\Property
				* 拓扑排序
					\Problem
						对有向无环图的点集的线性排序, 且满足
						- 每个点出现且只出现一次
						- 若存在一条从点 A 到点 B 的边,则在序列中点 A 出现在点 B 之前.

						\Note
							该问题能帮助判定某个图是否为有向无环图.

					\Algorithm
						- 迭代, 每次遍历所有边集, 删除入边数为0的点, 将其顺序加入目标序列, 直至所有点都从图中删除, 并加入目标序列. 若迭代至某次后, 图不再有点删除但剩余点数不为0, 则该图存在环路, 不为有向无环图.

			\Include
				* 链表
					\Define
						$
							V = \{v_i \ |\ i \in 1:n\}
							E = \{(v_i, v_{i+1}) \ |\ i \in 1:(n-1)\}  \tag{序号相邻两点连一条边}
						$
						形如: 
							$a \to b \to c \to d \to e \to ... \to x$

			\Property
				- $n$个点的链表有$n-1$条边.

