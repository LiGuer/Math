* 随机过程
	\def
		概率空间$Ω, \mathcal F, P$下, 从$Ω$ 到一个连续时间函数空间的映射$X(t, 	
		ζ)$.
		\Note
			- 时间固定时, 随机过程退化为随机变量.
			- 随机样本确定时, 随机过程退化为连续时间函数.
	\Property
		- 相关性
			不相关时, $R_{XY}(t_1, t_2) = μ_X(t_1) μ_Y^*(t_2), \qu \forall t_1, t_2$
		- 正交性
			$R_{XY}(t_1, t_2) = 0, \qu \forall t_1, t_2$
		- 独立性: 
		* 平稳性 (时间平移不变性)
			\def
				$\P(x_{t_1}, ... , x_{t_n}) = \P(x_{t_1+τ}, ..., x_{t_n+τ}) \qu \forall τ, t_1, ..., t_n \in \bb R, \qu n \in \bb N$
				统计特性不随随时间推移而改变. 
				\Note
					在时间序列上随意的任意间隔任意顺序的采样, 都具有时间平移不变性.
		* 广义平稳性
			* 功率谱密度
				\def
					功率谱密度, 是广义平稳过程的自相关函数的Fourier变换.
				\Property
					- 是实数
					- $≥ 0$

		* Markov性
	\Example
		- 简单过程
			$X(t, ζ) = X(ζ) f(t)$
		- 随机正弦波
			$X(t, ζ) = A(ζ) \sin(\omega_0 t + \Theta(ζ))$
		- Possion 计数过程
			$N(t) = \sum_{n=1}^\infty u(t - T(n)) \qu; f_T(t;n) = \/{(λ t)^{n-1}}{(n-1)!} λ e^{-λ t} u(t)$
