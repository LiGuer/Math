* 概率论
	* 概率空间
		\def
			概率空间是一个三元组 $<Ω, \mathcal F, \P>$. 
			- $Ω$ 样本空间; 
			- $\P$, 概率; 
			- $\mathcal F$ 被选择的样本集合的集合, 且满足
				- 包含空集、样本空间全集 $\emptyset, Ω \in \mathcal F$
				- 取补封闭, 如果一个事件A在其中, 那么补集也需要在其中.  
					$A \in \mathcal F => A^C \in \mathcal F$
				- 可列并封闭 
					$A_1, A_2, ... \in \mathcal F => \bigcup_{i=1}^∞ A_i \in \mathcal F$

			* 概率
				\def
					$\P: \mathcal F \to [0, 1]$
					概率是一种集合函数, 是一种对集合的测度, 且满足Kolmogorov公理.
					* Kolmogorov公理
						- 非负性 $\P(A) \in [0, 1] \qu ; \forall A \in F$
						- 规范性 $\P(Ω) = 1$
						- 可列可加性 $\P (\bigcup_i A_i) = \sum_i \P(A_i)$

				* 联合概率, 条件概率
					* 联合概率
						\def 
							.$\P(A B)$, $A B$一起发生的概率.
					* 条件概率
						\def
							.$\P(B | A)$, $A$发生条件下, $B$发生的概率.

					\Property
						* 独立性 $ <=> \P(A B) = \P(A) \P(B)$. 即, $A, B$的发生互不影响.
						- 联合概率 - 条件概率关系
							$
								\P(B | A) = \/{\P(A B)}{\P(A)}
								\P(A B) = \P(B | A) \P(A) = \P(A | B) \P(B)
							$
						* \Theorem{全概率公式}
							$\P(A) = \sum_i \P(A|B_i) \P(B_i) \qu; \sum_i A_i = Ω$
						* \Theorem{Bayes公式}
							$
								\P(A | B) = \/{\P(B | A) \P(A)}{\P(B)}
								\P(A_i | B) = \/{\P(B | A_i) \P(A_i)}{\sum_j \P(B|A_j) \P(A_j)}; \sum_j A_j = Ω
							$

				\Theorem{大数定律}
					* 弱大数定理
						$\lim_{n \to ∞} \P(|1/N \sum_{k=1}^n X_k-μ|<ε)=1$
					* Bernoulli大数定理
						$\lim_{n \to ∞} \P(|\/{f_A}{n}-p|<ε) = 1$
				\Theorem{中心极限定律}
					$\lim_{n \to∞} F_n(x) =\lim_{n \to∞} \P(\/{\sum_{k=1}^n X_k - n μ}{\sqrt{n} σ} ≤ x)=\int_{-∞}^x \/{1}{\sqrt{2 π}} e^{-t^2 / 2} \d t=\Phi(x)$
					
	* 随机变量
	* 随机过程
